\documentclass[12pt,a4paper]{scrartcl}
\usepackage[utf8]{inputenc}
\usepackage[T1]{fontenc}
\usepackage[french]{babel}
\usepackage{textcomp}
\usepackage{mathtools,amssymb}
\usepackage{lmodern}
\usepackage{graphicx}
\usepackage[dvipsnames,svgnames]{xcolor}
\usepackage{microtype}
\usepackage{hyperref} 
\usepackage{tikz}
\usepackage{float}

\hypersetup{
	colorlinks=true,
	linkcolor=Brown,
	urlcolor=Navy, 
	breaklinks=true,
	bookmarks=true,
	pdfstartview=XYZ
}
\usepackage{colortbl, soul}

\usepackage{listings} 
\lstset{
	inputencoding=utf8,
	extendedchars=true,
	literate={é}{{\'e}}1 {è}{{\`e}}1 {à}{{\`a}}1,
	upquote = true,
	columns = flexible,
	basicstyle = \ttfamily,
	keepspaces=true,
	language = python,
	backgroundcolor = \color{Navy!10}, keywordstyle = \color{ForestGreen}\bfseries, % numbers=left, numberstyle=\tiny,
	tabsize = 4,
}

\usepackage{amsthm}

\theoremstyle{plain} 
\newtheorem{theoreme}{Théorème}[section] 
%\newtheorem{proposition}[theoreme]{Proposition} 
%\newtheorem{corollaire}[theoreme]{Corollaire} 
%\newtheorem{lemme}[theoreme]{Lemme}
%\newtheorem{exo}{Exercice}
\theoremstyle{definition} 
\newtheorem{definition}[theoreme]{Définition}
\theoremstyle{remark} 
\newtheorem*{remarque}{Remarque}


\newcommand{\cad}{\text{c'est à dire }}
\newcommand{\guillemets}[1]{\og #1\fg} % Commande Mme Gloria Faccanoni

%\title{Rapport Projet I62} \author{Ivan CROS, Matteo BONAVITA, Clément MAYNADIER - Groupe A1} \date{\today}

\newcommand{\monuniversite}{Université de Toulon, I62 Génie Logiciel}
\newcommand{\autheur}{Ivan CROS, Matteo BONAVITA, Clément MAYNADIER - Groupe A1}
\newcommand{\titrerapport}{Rapport Projet I62}


\usepackage{fancyhdr}

\fancyhead[L]{\leftmark}
\fancyhead[C]{}
\fancyhead[R]{}

\fancyfoot[L]{}
\fancyfoot[C]{}
\fancyfoot[R]{\thepage}

\pagestyle{fancy}

\begin{document}
	
	\begin{titlepage}
		\begin{tikzpicture}[remember picture,overlay]
		        \shade[top color=blue!70,bottom color=blue!10] (current page.north west) rectangle (current page.south east);
		\end{tikzpicture}
		
		\centering
		\vspace*{2cm}
		{\scshape\Large \textcolor{white}{\monuniversite} \par}
		\vspace{1.5cm}
		{\huge\bfseries \textcolor{white}{\titrerapport} \par}
		
		\vspace{1.5cm}
		{\Large\itshape \textcolor{white}{\autheur} \par}
			\vspace{2cm}
			
		\begin{figure}[h]
			\centering
			\includegraphics[height=8cm]{img/Virginia_class_submarine.jpg} 
			\caption{Illustration d'un sous-marin d'attaque, réalisé par Ron Stern}
		\end{figure}
		
		
	
		
		\vfill
		{\large  \par}
	\end{titlepage}
	
	%	Page de garde
	%\maketitle
	\section*{Objectif du rapport}
	\begin{abstract}
	Afin de garantir la viabilité du projet avant d'investir dans des drones réels, notre client a fait l'acquisition d'un simulateur de drones. Le but est d'augmenter la vision des sous-marins d’attaque dans leurs environnement par l’intermédiaire de drones autonomes aériens.  Le développement du produit nécessite une collaboration avec nos clients pour permettre notre équipe de s’initier au domaine naval. Pour que notre équipe puisse se familiariser au domaine naval, il est essentiel de travailler en étroite collaboration avec nos clients tout au long du développement du produit. L'objectif de se rapport est de documenter le produit livrable selon l'analyse des besoins pour faciliter sa maintenance et son déploiement.  
\end{abstract}
	
	
	%fin page de garde 
	%\newpage
	
	%Sommaire
	\tableofcontents
	
	
	
	
	\section{Présentation des membres du groupe}
	
	\begin{figure}[h]
		\centering
		\includegraphics[height=8cm]{img/diagCollaboration.png} 
		\caption{Illustration de l'attribution des rôles principaux au sein de l'équipe.}
	\end{figure}
	
	
	
	\section{Exigences fonctionnelles (listes par catégorie)}
	
\subsection{Interface}
\begin{itemize}
	\item Le Logiciel doit être séparé entre un serveur et des clients
\end{itemize}

%	\hrulefill

\subsection{Serveur}
\begin{itemize}
	\item Le serveur doit posséder un éditeur de situation (EdS).
	\item Le serveur doit simuler la situation
	\item Le serveur doit offrir une vue 'replay' des résultat clients
\end{itemize}



\subsection{Editeur de situation (EdS)}
\begin{itemize}
	\item L'EdS doit permettre de créer ou modifier une situation.
	\item L'EdS doit permettre d'ajouter des obstacles OPFOR\footnote{Toute force ennemie: drônes, navires} à une situation.
	\item L'EdS doit permettre de définir une carte topographique de la situation.
	\item L'EdS doit permettre de définir un ou plusieurs points de départ pour les sous-marins de la situation.
	\item L'EdS doit permettre de définir un ou plusieurs objectifs à la situation.
\end{itemize}

\subsection{Une situation}
\begin{itemize}
	\item La situation doit pouvoir contenir des unités OPFOR.
	\item La situation doit contenir une carte topographique.
	\item La situation doit contenir un ou plusieurs points de départ pour les sous-marins de la situation.
	\item La situation doit contenir un ou plusieurs objectifs.
\end{itemize}


\subsection{Simulation}

Chaque objectif doit être assigné à un sous-marin uniquement.

\begin{itemize}
	\item La simulation doit être jouée sur le serveur.
	\item La simulation doit avoir ses sous-marins contrôlés par les clients.
\end{itemize}

%	\hrulefill

\subsection{OPFOR (forces ennemies)}
\begin{itemize}
	\item L'OPFOR doit avoir des unités anti-aériennes.
	\item L'OPFOR doit avoir des unités navales capables de se déplacer durant la situation.
	\item L'OPFOR doit avoir des unités autres pour les objectifs de reconnaissance.
\end{itemize}
%	\hrulefill

\subsection{Les unités}
\begin{itemize}
	\item Les unités AA OPFOR doivent pouvoir détecter les drones BLUFOR\footnote{Tout ce qui est unité allié :  les sous-marins, les drones...}.
	\item Les unités AA OPFOR doivent pouvoir détruire les drones BLUFOR.
\end{itemize}


\subsection{La vision}	
\begin{itemize}
	\item La vision replay doit offrir une vue BLUFOR de la situation.
	\item La vision replay doit offrir une vue omnisciente de la situation.
\end{itemize}

%	\hrulefill

\subsection{Le client}
Concernant le client : 
\begin{itemize}
	\item Le client doit planifier une résolution de la situation pour son sous-marin.
	\item Le client doit gérer un sous-marin et sa flotte de drones.
	\item Le client doit pouvoir contrôler le sous-marin.
	\item Le client doit afficher ce que son sous-marin voit.
	\item Le client doit afficher ce que ses drones à portée voient.
\end{itemize}

%\hrulefill

\subsection{Les drones}
\begin{itemize}
	\item Les drones doivent pouvoir patrouiller autour du sous-marin.
	\item Les drones doivent pouvoir faire de la reconnaissance.
	\item Les drones doivent pouvoir faire des missions de reconnaissance autonomes.
	\item Les drones doivent pouvoir faire garde en autonomie.
	\item Les drones doivent pouvoir être contrôlés par le client.
	\item Les drones doivent envoyer des rapports de détection à la base de données.
\end{itemize}

La base de données doit contenir des informations sur les drones.

\subsection{Les sous-marins}
\begin{itemize}
	\item Le sous-marin doit pouvoir changer de profondeur.
	\item Le sous-marin doit pouvoir communiquer avec ses drones.
	\item Le sous-marin doit pouvoir communiquer avec son commandement.
\end{itemize}

	
	\section{Maquettage du produit}
	\subsection{Serveur}
		\begin{figure}[H]
		\centering
		\includegraphics[height=10cm]{img/maquette/serveur/1.png} 
	\end{figure}
		\begin{figure}[H]
		\centering
		\includegraphics[height=10cm]{img/maquette/serveur/2.png} 
	\end{figure}
		\begin{figure}[H]
		\centering
		\includegraphics[height=10cm]{img/maquette/serveur/3.png} 
	\end{figure}
		\begin{figure}[H]
		\centering
		\includegraphics[height=10cm]{img/maquette/serveur/4.png} 
	\end{figure}
		\begin{figure}[H]
		\centering
		\includegraphics[height=10cm]{img/maquette/serveur/5.png} 
	\end{figure}
		\begin{figure}[H]
		\centering
		\includegraphics[height=10cm]{img/maquette/serveur/6.png} 
	\end{figure}
		\begin{figure}[H]
		\centering
		\includegraphics[height=10cm]{img/maquette/serveur/7.png} 
	\end{figure}
		\begin{figure}[H]
		\centering
		\includegraphics[height=10cm]{img/maquette/serveur/8.png} 
	\end{figure}
	
		\begin{figure}[H]
		\centering
		\includegraphics[height=10cm]{img/maquette/serveur/final.png} 
			\caption{Rendu final attendu (illustration sitac reprise du \href{https://fr.pilotaware.com/post/strategic-vs-tactical-awareness}{PilotAware Rosetta})}
	\end{figure}
	
		\subsection{Client}
	
		\begin{figure}[H]
		\centering
		\includegraphics[height=10cm]{img/maquette/Client/ihm_client1.png} 
	\end{figure}
			\begin{figure}[H]
		\centering
		\includegraphics[height=10cm]{img/maquette/Client/ihm_client2.png} 
	\end{figure}
			\begin{figure}[H]
		\centering
		\includegraphics[height=10cm]{img/maquette/Client/ihm_client3.png} 
	\end{figure}
	
		\begin{figure}[H]
		\centering
		\includegraphics[height=10cm]{img/maquette/Client/final.png} 
		\caption{Rendu final attendu (illustration sitac reprise du \href{https://fr.pilotaware.com/post/strategic-vs-tactical-awareness}{PilotAware Rosetta})}
	\end{figure}
	
	
	\section{Conception UML}
		\subsection{Axe fonctionnel}



\subsubsection{Diagramme de cas d'usage SI Serveur}
\begin{figure}[H]
	\centering
	\includegraphics[height=8cm]{img/CUSI_Serveur.png} 
	\caption{CU SI Serveur.}
\end{figure}
\subsubsection{Diagramme de cas d'usage SI Client}
\begin{figure}[H]
	\centering
	\includegraphics[height=13cm]{img/CUSI_Client.png} 
	\caption{CU SI Client.}
\end{figure}


\subsubsection{Structure du système niveau 1}
\begin{figure}[H]
	\centering
	\includegraphics[height=5cm]{img/DC_L1Client.png} 
	\caption{Structure pour le logiciel Client}
\end{figure}

\begin{figure}[H]
	\centering
	\includegraphics[height=5cm]{img/DC_L1Serveur.png} 
	\caption{Structure pour le logiciel Serveur.}
\end{figure}


\subsubsection{Structure du système niveau 2}
\begin{figure}[H]
	\centering
	\includegraphics[height=15cm]{img/DC_L2_Serveur.png} 
	\caption{Partie serveur}
\end{figure}

\begin{figure}[H]
	\centering
	\includegraphics[height=15cm]{img/DC_L2_Client.png} 
	\caption{Partie Client}
\end{figure}



\subsection{Axe statique}
\subsubsection{Diagramme des classes partie Serveur}
\begin{figure}[H]
	\centering
	\includegraphics[height=15cm]{img/serveur.png} 
	\caption{Diagramme des classes pour le simulateur partie Serveur}
\end{figure}

\subsubsection{Diagramme des classes partie Client}
\begin{figure}[H]
	\centering
	\includegraphics[height=15cm]{img/client.png} 
	\caption{Diagramme des classes pour le simulateur partie Client.}
\end{figure}


\subsection{Axe dynamique}
\subsubsection{Diagramme de séquence Serveur}

\begin{figure}[H]
	\centering
	\includegraphics[height=8cm]{img/SeqServeur.png} 
	\caption{Diag. de séq. sur l'intéraction entre le serveur et ses différents composants lors de la configuration d'une simulation.}
\end{figure}

\begin{figure}[H]
	\centering
	\includegraphics[height=8cm]{img/ServeurRole.png} 
	\caption{Diag. de séq. sur l'attribution des rôles aux clients par le serveur.}
\end{figure}

\subsubsection{Diagramme de séquence Client}
\begin{figure}[H]
	\centering
	\includegraphics[height=8cm]{img/DiagSeq_IntenvoiRapport.png} 
	\caption{Diag. de séq. pour l'envoi de rapport par le Client}
\end{figure}

\begin{figure}[H]
	\centering
	\includegraphics[height=8cm]{img/diagSeqModif.png} 
	\caption{Diag. de séq. pour l'envoi des modifications }
\end{figure}


\subsubsection{Base de donnée: Modèle Entité-Association}
\begin{figure}[H]
	\centering
	\includegraphics[height=8cm]{img/MEA.png} 
	\caption{MEA pour la BD de Serveur}
\end{figure}


	
	\section{Le manuel d’installation du logiciel}
	 %\input{file}
	 \subsection{Installation des bibliothèques}
	 
	 \subsection{Tkinter}
	 Tkinter est une bibliothèque pour créer des interfaces graphiques en Python. Si vous utilisez Python 3, il est probablement déjà installé. Sinon, vous pouvez l'installer en utilisant pip :
	 \begin{verbatim}
	 	pip install tk
	 \end{verbatim}
	 
	 \subsection{Pillow}
	 Pillow est une bibliothèque de traitement d'images en Python. Vous pouvez l'installer via pip :
	 \begin{verbatim}
	 	pip install Pillow
	 \end{verbatim}
	 


\subsection{Comment procéder à l'exécution}

Pour le logiciel de configuration de simulation côté serveur :

\begin{enumerate}
	\item Placez-vous dans le répertoire Code/server.
	\item Ouvrez un terminal.
	\item Exécutez le fichier client.py en utilisant la commande suivante :
	\begin{verbatim}
		python3 client.py
	\end{verbatim}
\end{enumerate}

Pour le logiciel de simulation côté client :

\begin{enumerate}
	\item Placez-vous dans le répertoire Code/client.
	\item Ouvrez un terminal.
	\item Exécutez le fichier serveur.py en utilisant la commande suivante :
	\begin{verbatim}
		python3 serveur.py
	\end{verbatim}
\end{enumerate}



	 \section{Le manuel utilisateur du logiciel}
	 %\input{file}
	  \subsection{Application Client}
	 
	 L'application côté client permet trois actions différentes : "Envoyer drônes", "Gérer sous-marin" et "Générer un rapport".
	 
	 \begin{itemize}
	 	\item "Envoyer drônes" ordonne à un de vos drônes de commencer une opération. Vous pouvez sélectionner plusieurs drônes à la fois. L'application indiquera combien de drônes sont en opération et combien sont disponibles. Lorsque vous envoyez un drône ou plusieurs en opération, vous pouvez définir leur trajectoire, les diviser en plusieurs groupes et choisir un mode d'action entre faire de la reconnaissance, explorer ou faire de l'écoute.
	 	\item "Gérer sous-marin" permet de donner un cap au sous-marin. Vous pouvez aussi donner l'ordre d'écouter ou d'attaquer un autre sous-marin, ou de blanchir (sécuriser) une zone. Pour blanchir une zone, choisissez trois points de l'espace, ces trois points définiront la zone triangulaire à blanchir.
	 	\item "Générer un rapport" permet d'envoyer un rapport au serveur en choisissant les données à fournir. Vous pouvez consulter le rapport à tout moment.
	 \end{itemize}
	 
	 La partie droite de l'application représente la carte avec au centre le sous-marin contrôlé. Les drônes sous le contrôle du sous-marin sont représentés par des cônes, les unités alliées en vert et les OPFOR en rouge.
	 
	 \subsection{Application Serveur}
	 
	 L'application côté serveur permet de faire les actions suivantes :
	 
	 \begin{itemize}
	 	\item Consulter un rapport : entrez le numéro d'une mission passée pour pouvoir consulter son rapport.
	 	\item Créer une unité : vous permet de créer un drône ou un navire qui sera l'acteur principal d'une simulation. Après avoir créé l'unité, cliquez sur "Créer une simulation".
	 	\item Gérer une simulation : après avoir créé l'acteur principal, vous permet de définir une simulation et de la lancer. Avant de pouvoir lancer la simulation, il est nécessaire de compléter toutes les étapes de la création, à savoir placer les unités de la simulation, définir le terrain, le type de simulation, l'OPFOR et définir un client pour cette simulation.
	 	\item Visualisation : affiche le déroulement de la simulation une fois lancée.
	 \end{itemize}
	 
	 
	 \section{Le planning mis en œuvre durant le projet}
	 	\begin{figure}[h]
	 	\centering
	 	\includegraphics[height=7cm]{img/Gant.png} 
	 	\caption{Diagramme de Gantt pour la planification du projet.}
	 \end{figure}
	 

\subsection{Sprint 1}

		\subsubsection{Sprint Planning  | 27 février 2024}
		\begin{enumerate}
			\item Liste de l’ensemble des demandes du client.
			\item Cadrage initial du projet et définition des objectifs; pour se faire : réflexion avec l'équipe et le représentant du client pour se familiariser avec le domaine du produit.
		\end{enumerate}
		
		\subsubsection*{Tâches à réaliser}
		\begin{itemize}
			\item Définition des besoins : surveillance, renseignement, reconnaissance.
			\item Schématisation du navires et de ses drones.
			\item Familiarisation avec la doctrine detection passive.
			\item Prise en compte de la dimension distribution serveur/client pour simuler les échanges entre les systèmes.
			\item Prototype papier d’une base de données pour contenir les rapports de détection, les caractéristiques des drones et l’état des missions.
		\end{itemize}
		
		\subsubsection{Daily Scrum du 12 Mars 2024}
		\subsubsection*{Objectif(s)}
		\begin{itemize}
			\item Liste de l’ensemble des demandes et prise en compte des retours du client.
		\end{itemize}
		
		\subsubsection*{Tâches réalisées}
		\begin{itemize}
			\item Listes des exigences fonctionnelles.
			\item Échange entre le client, son représentant pour éclaircir les zones de flou sur le fonctionnement du système et lever des barrières et des interrogations autour des différents usages du système à travers d’un schéma et d’échanges.
		\end{itemize}



\subsection{Sprint 2}
	
	\subsubsection{Sprint planning | 19 Mars 2024}
	\subsubsection*{Objectif(s)}
	\begin{itemize}
		\item Définitions des objectifs à réaliser entre l’équipe et les représentants du client.
		\item Planification des tâches.
	\end{itemize}
	
	\subsubsection*{Tâches réalisées}
	\begin{itemize}
		\item Réalisation du diagramme de cas d’usage.
		\item Liste des différentes fonctions pour chaque mode du simulateur.
		\item Définition des contraintes du système et de son articulation.
	\end{itemize}
	
	\subsubsection{Sprint Review du 26 mars 2024}
	
	\subsubsection*{Tâches réalisées}
	\begin{itemize}
		\item Réalisation du diagramme statique (objet).
		\item Assimilation des rôles pour la méthode Agile Scrum.
		\item Recherche de développement du code : PyQT ou Tkinter.
	\end{itemize}
	
	\subsubsection{Daily Scrum | 8 Avril 2024}
	\subsubsection*{Objectif(s)}
	\begin{itemize}
		\item Clarification de ce qui a été réalisé, des différentes tâches à finaliser et démarrage d’un cycle extreme programming.
	\end{itemize}
	
	\subsubsection*{Tâches réalisées}
	\begin{itemize}
		\item Échange au tableau sur la conception IHM de notre simulateur.
		\item Développement en binôme : Clément pour la partie serveur et Mattéo pour la maquette de l’interface. Programmation modulaire avec Tkinter en Python3.
		\item Réalisation et reprise des diagrammes UML.
	\end{itemize}
	
	\subsubsection{Sprint Review | 16 Avril 2024}
	\subsubsection*{Objectif(s)}
	\begin{itemize}
		\item Vérification du code et remise en question du développement.
	\end{itemize}
	
	\subsubsection*{Tâches réalisées}
	\begin{itemize}
		\item Ajustement de l’IHM à partir de la maquette.
		\item Échange avec le client pour réviser l’approche client-serveur.
	\end{itemize}
	
	\subsubsection{Sprint Retrospective 1 | 23 Avril 2024}
	\subsubsection*{Objectif(s)}
	\begin{itemize}
		\item Clarification de ce qui a été réalisé.
		\item Reprise des diagrammes UML.
		\item Développement des classes selon le MVC.
	\end{itemize}
	
	\subsubsection{Sprint Retrospective 2 | 29 Avril 2024}
	\subsubsection*{Clarifications}
	\begin{itemize}
		\item Briefing sur ce qui a été réalisé.
		\item Réalisation d'un diagramme de Gantt.
		\item Réalisation d'un diagramme de séquence dynamique
		\item Vérification du code selon le modèle UML.
	\end{itemize}
	
	



\section{Diagramme de Gantt}

\subsection*{29 Avril 2024}
\begin{figure}[H]
	\centering
	\includegraphics[height=8cm]{img/gantt_290424.png}
	\caption{Diagramme de Gantt le 29 Avril 2024}
\end{figure}

\begin{table}[H]
	\centering
	\caption{Planning du Projet}
	\begin{tabular}{
			>{\columncolor[HTML]{C0C0C0}}l |r|r|r|r|}
		\cline{2-5}
		\cellcolor[HTML]{EFEFEF}                                             & \multicolumn{1}{l|}{\cellcolor[HTML]{C0C0C0}Date de début} & \multicolumn{1}{l|}{\cellcolor[HTML]{C0C0C0}Jours passées} & \multicolumn{1}{l|}{\cellcolor[HTML]{C0C0C0}Jours restants} & \multicolumn{1}{l|}{\cellcolor[HTML]{C0C0C0}TOTAL} \\ \hline
		\multicolumn{1}{|l|}{\cellcolor[HTML]{C0C0C0}Échange avec le client} & 27/02/2024                                                 & 3                                                          & 0                                                           & 01/03/2024                                         \\ \hline
		\multicolumn{1}{|l|}{\cellcolor[HTML]{C0C0C0}Sprints}                & 27/02/2024                                                 & 7                                                          & 14                                                          & 19/03/2024                                         \\ \hline
		\multicolumn{1}{|l|}{\cellcolor[HTML]{C0C0C0}Axe Fonctionnel}        & 23/04/2024                                                 & 4                                                          & 0                                                           & 27/04/2024                                         \\ \hline
		\multicolumn{1}{|l|}{\cellcolor[HTML]{C0C0C0}Axe Statique}           & 19/03/2024                                                 & 3                                                          & 0                                                           & 22/03/2024                                         \\ \hline
		\multicolumn{1}{|l|}{\cellcolor[HTML]{C0C0C0}Axe Dynamique}          & 29/04/2024                                                 & 0                                                          & 14                                                          & 13/05/2024                                         \\ \hline
		\multicolumn{1}{|l|}{\cellcolor[HTML]{C0C0C0}Code Python}            & 29/04/2024                                                 & 3                                                          & 12                                                          & 14/05/2024                                         \\ \hline
		\multicolumn{1}{|l|}{\cellcolor[HTML]{C0C0C0}Rédaction}              & 05/03/2024                                                 & 11                                                         & 9                                                           & 15/05/2024                                         \\ \hline
	\end{tabular}
\end{table}

\subsection*{8 Mai 2024}
\begin{figure}[H]
	\centering
	\includegraphics[height=8cm]{img/gantt_08052024.png}
	\caption{Diagramme de Gantt le 8 Mai 2024}
\end{figure}

% Please add the following required packages to your document preamble:
% \usepackage[table,xcdraw]{xcolor}
% Beamer presentation requires \usepackage{colortbl} instead of \usepackage[table,xcdraw]{xcolor}
% Please add the following required packages to your document preamble:
% \usepackage[table,xcdraw]{xcolor}
% Beamer presentation requires \usepackage{colortbl} instead of \usepackage[table,xcdraw]{xcolor}
\begin{table}[H]
	\begin{tabular}{
			>{\columncolor[HTML]{EFEFEF}}l |r|r|r|r|}
		\cline{2-5}
		& \multicolumn{1}{l|}{\cellcolor[HTML]{EFEFEF}Date de début} & \multicolumn{1}{l|}{\cellcolor[HTML]{EFEFEF}Jours passées} & \multicolumn{1}{l|}{\cellcolor[HTML]{EFEFEF}Jours restants} & \multicolumn{1}{l|}{\cellcolor[HTML]{EFEFEF}TOTAL} \\ \hline
		\multicolumn{1}{|l|}{\cellcolor[HTML]{EFEFEF}Échange avec le client} & 27/02/2024                                                 & 4                                                          & 0                                                           & 01/03/2024                                         \\ \hline
		\multicolumn{1}{|l|}{\cellcolor[HTML]{EFEFEF}Sprints}                & 27/02/2024                                                 & 7                                                          & 5                                                           & 19/03/2024                                         \\ \hline
		\multicolumn{1}{|l|}{\cellcolor[HTML]{EFEFEF}Axe Fonctionnel}        & 23/04/2024                                                 & 4                                                          & 0                                                           & 27/04/2024                                         \\ \hline
		\multicolumn{1}{|l|}{\cellcolor[HTML]{EFEFEF}Axe Statique}           & 19/03/2024                                                 & 3                                                          & 0                                                           & 22/03/2024                                         \\ \hline
		\multicolumn{1}{|l|}{\cellcolor[HTML]{EFEFEF}Axe Dynamique}          & 29/04/2024                                                 & 0                                                          & 0                                                           & 13/05/2024                                         \\ \hline
		\multicolumn{1}{|l|}{\cellcolor[HTML]{EFEFEF}Code Python}            & 29/04/2024                                                 & 4                                                          & 9                                                           & 17/05/2024                                         \\ \hline
		\multicolumn{1}{|l|}{\cellcolor[HTML]{EFEFEF}Rédaction}              & 05/03/2024                                                 & 12                                                         & 9                                                           & 17/05/2024                                         \\ \hline
	\end{tabular}
\end{table}

\subsection*{Bilan de la méthode Agile}
Nous avons pu rectifier le livrable que nous allons fournir en un seul logiciel, en deux logiciels distincts (client et serveur). Ces améliorations ont été discutées lors des échanges, tout comme la création d’une interface graphique pour la partie client alors que nous pensions plutôt à une interaction par commande via un terminal.  

	 
	
	
\end{document}